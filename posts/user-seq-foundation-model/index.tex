% Options for packages loaded elsewhere
% Options for packages loaded elsewhere
\PassOptionsToPackage{unicode}{hyperref}
\PassOptionsToPackage{hyphens}{url}
\PassOptionsToPackage{dvipsnames,svgnames,x11names}{xcolor}
%
\documentclass[
  letterpaper,
  DIV=11,
  numbers=noendperiod]{scrartcl}
\usepackage{xcolor}
\usepackage{amsmath,amssymb}
\setcounter{secnumdepth}{-\maxdimen} % remove section numbering
\usepackage{iftex}
\ifPDFTeX
  \usepackage[T1]{fontenc}
  \usepackage[utf8]{inputenc}
  \usepackage{textcomp} % provide euro and other symbols
\else % if luatex or xetex
  \usepackage{unicode-math} % this also loads fontspec
  \defaultfontfeatures{Scale=MatchLowercase}
  \defaultfontfeatures[\rmfamily]{Ligatures=TeX,Scale=1}
\fi
\usepackage{lmodern}
\ifPDFTeX\else
  % xetex/luatex font selection
\fi
% Use upquote if available, for straight quotes in verbatim environments
\IfFileExists{upquote.sty}{\usepackage{upquote}}{}
\IfFileExists{microtype.sty}{% use microtype if available
  \usepackage[]{microtype}
  \UseMicrotypeSet[protrusion]{basicmath} % disable protrusion for tt fonts
}{}
\makeatletter
\@ifundefined{KOMAClassName}{% if non-KOMA class
  \IfFileExists{parskip.sty}{%
    \usepackage{parskip}
  }{% else
    \setlength{\parindent}{0pt}
    \setlength{\parskip}{6pt plus 2pt minus 1pt}}
}{% if KOMA class
  \KOMAoptions{parskip=half}}
\makeatother
% Make \paragraph and \subparagraph free-standing
\makeatletter
\ifx\paragraph\undefined\else
  \let\oldparagraph\paragraph
  \renewcommand{\paragraph}{
    \@ifstar
      \xxxParagraphStar
      \xxxParagraphNoStar
  }
  \newcommand{\xxxParagraphStar}[1]{\oldparagraph*{#1}\mbox{}}
  \newcommand{\xxxParagraphNoStar}[1]{\oldparagraph{#1}\mbox{}}
\fi
\ifx\subparagraph\undefined\else
  \let\oldsubparagraph\subparagraph
  \renewcommand{\subparagraph}{
    \@ifstar
      \xxxSubParagraphStar
      \xxxSubParagraphNoStar
  }
  \newcommand{\xxxSubParagraphStar}[1]{\oldsubparagraph*{#1}\mbox{}}
  \newcommand{\xxxSubParagraphNoStar}[1]{\oldsubparagraph{#1}\mbox{}}
\fi
\makeatother


\usepackage{longtable,booktabs,array}
\usepackage{calc} % for calculating minipage widths
% Correct order of tables after \paragraph or \subparagraph
\usepackage{etoolbox}
\makeatletter
\patchcmd\longtable{\par}{\if@noskipsec\mbox{}\fi\par}{}{}
\makeatother
% Allow footnotes in longtable head/foot
\IfFileExists{footnotehyper.sty}{\usepackage{footnotehyper}}{\usepackage{footnote}}
\makesavenoteenv{longtable}
\usepackage{graphicx}
\makeatletter
\newsavebox\pandoc@box
\newcommand*\pandocbounded[1]{% scales image to fit in text height/width
  \sbox\pandoc@box{#1}%
  \Gscale@div\@tempa{\textheight}{\dimexpr\ht\pandoc@box+\dp\pandoc@box\relax}%
  \Gscale@div\@tempb{\linewidth}{\wd\pandoc@box}%
  \ifdim\@tempb\p@<\@tempa\p@\let\@tempa\@tempb\fi% select the smaller of both
  \ifdim\@tempa\p@<\p@\scalebox{\@tempa}{\usebox\pandoc@box}%
  \else\usebox{\pandoc@box}%
  \fi%
}
% Set default figure placement to htbp
\def\fps@figure{htbp}
\makeatother





\setlength{\emergencystretch}{3em} % prevent overfull lines

\providecommand{\tightlist}{%
  \setlength{\itemsep}{0pt}\setlength{\parskip}{0pt}}



 


\KOMAoption{captions}{tableheading}
\makeatletter
\@ifpackageloaded{caption}{}{\usepackage{caption}}
\AtBeginDocument{%
\ifdefined\contentsname
  \renewcommand*\contentsname{Table of contents}
\else
  \newcommand\contentsname{Table of contents}
\fi
\ifdefined\listfigurename
  \renewcommand*\listfigurename{List of Figures}
\else
  \newcommand\listfigurename{List of Figures}
\fi
\ifdefined\listtablename
  \renewcommand*\listtablename{List of Tables}
\else
  \newcommand\listtablename{List of Tables}
\fi
\ifdefined\figurename
  \renewcommand*\figurename{Figure}
\else
  \newcommand\figurename{Figure}
\fi
\ifdefined\tablename
  \renewcommand*\tablename{Table}
\else
  \newcommand\tablename{Table}
\fi
}
\@ifpackageloaded{float}{}{\usepackage{float}}
\floatstyle{ruled}
\@ifundefined{c@chapter}{\newfloat{codelisting}{h}{lop}}{\newfloat{codelisting}{h}{lop}[chapter]}
\floatname{codelisting}{Listing}
\newcommand*\listoflistings{\listof{codelisting}{List of Listings}}
\makeatother
\makeatletter
\makeatother
\makeatletter
\@ifpackageloaded{caption}{}{\usepackage{caption}}
\@ifpackageloaded{subcaption}{}{\usepackage{subcaption}}
\makeatother
\usepackage{bookmark}
\IfFileExists{xurl.sty}{\usepackage{xurl}}{} % add URL line breaks if available
\urlstyle{same}
\hypersetup{
  pdftitle={PinFM \& Approaches to User-Sequence Modeling},
  pdfauthor={Zhong Zhang},
  colorlinks=true,
  linkcolor={blue},
  filecolor={Maroon},
  citecolor={Blue},
  urlcolor={Blue},
  pdfcreator={LaTeX via pandoc}}


\title{PinFM \& Approaches to User-Sequence Modeling}
\author{Zhong Zhang}
\date{2025-10-01}
\begin{document}
\maketitle


\section{Overview}\label{overview}

There are three main approaches to user-sequence modeling:

\begin{enumerate}
\def\labelenumi{\arabic{enumi}.}
\tightlist
\item
  \textbf{Pure sequential models} that treat the clickstream as a
  standalone sequence and learn end-to-end (e.g., Meta's HSTU).
\item
  \textbf{Sequence encoders inside existing recommender architectures},
  where a transformer encodes the user's activity sequence and its
  representation is fed to the ranker (e.g., TransAct, TWIN).
\item
  \textbf{Pretrain-then-fine-tune foundation models}, where a
  sequence-only model is pretrained across applications on large-scale
  user activity and then fine-tuned alongside task-specific features in
  downstream models.
\end{enumerate}

This note focuses on \textbf{PinFM}, a representative of approach (3).

\begin{center}\rule{0.5\linewidth}{0.5pt}\end{center}

\section{PinFM: A Foundation Model for User Activity
Sequences}\label{pinfm-a-foundation-model-for-user-activity-sequences}

\textbf{Setting.} A billion-scale visual discovery platform trains a
large transformer (\textbf{20B+ parameters}) on two years of user
actions, then plugs it into multiple downstream ranking stacks (e.g.,
Home Feed, Item-to-Item). The aim is to \textbf{reuse one strong
sequence encoder across apps} under \textbf{hard latency/cost
constraints}.

\subsection{Pretraining
(sequence-only)}\label{pretraining-sequence-only}

\textbf{Inputs.} ID-based tokens:

\begin{itemize}
\tightlist
\item
  Item IDs, surface types, and action types are embedded.
\item
  Tokens pass through a small input MLP with L2-norm (
  \phi\_\{\text{in}\} ).
\item
  A transformer backbone produces hidden states.
\item
  An output MLP with L2-norm ( \phi\_\{\text{out}\} ) yields a sequence
  of user states ( H = (H\_1,\dots,H\_L) ).
\end{itemize}

\begin{quote}
Observation: Unlike many ID-based ranking models, \textbf{pretraining
does not suffer from one-epoch overfitting}.
\end{quote}

\subsubsection{Contrastive objective
(InfoNCE)}\label{contrastive-objective-infonce}

For state ( H\_i ), the positive is the next positively engaged item
embedding ( z\_\{i+1\} = \mathrm{emb}(\mathrm{id}\_\{i+1\}) ). Negatives
are in-batch item embeddings excluding items positively engaged by the
same user. The temperature ( \tau ) is learned.

We minimize \[
\mathcal{L}_{\text{InfoNCE}}(i)
= - \log
\frac{\exp\!\big(\mathrm{sim}(H_i, z_{i+1})/\tau\big)}
{\sum\limits_{j \in \mathcal{N}_i}\exp\!\big(\mathrm{sim}(H_i, z_j)/\tau\big)} \, ,
\] which scales to large vocabularies without a full softmax.

\subsubsection{Three stacked losses (serve-time
aligned)}\label{three-stacked-losses-serve-time-aligned}

To better match user behavior and serving constraints, pretraining
stacks three InfoNCE-style losses:

\begin{itemize}
\tightlist
\item
  \textbf{NTL (Next-Token Loss):} apply only when the next token is a
  \textbf{positive}.
\item
  \textbf{MTL (Multi-Token Loss):} predict \textbf{all positives in a
  short future window} to capture short-term (``sticky'') interests.
\item
  \textbf{FTL (Future-Token Loss):} emphasize the state near serve-time
  length ( L\_d ) by predicting a \textbf{future window from (
  H\_\{L\_d\} )} (instruction-tuning-style alignment).
\end{itemize}

\textbf{Serve-time alignment example.} Pretraining segments use ( L =
100 ), but serving can afford only ( L\_d = 20 ). Compute (
(H\_1,\dots,H\_\{100\}) ), take ( H\_\{20\} ), and apply InfoNCE to
predict positives in a short window (e.g., steps (21!\dots!25)).

\begin{center}\rule{0.5\linewidth}{0.5pt}\end{center}

\subsection{Fine-tuning (plugging into
rankers)}\label{fine-tuning-plugging-into-rankers}

PinFM is added as a \textbf{user-sequence module} inside DLRM/DCN-style
stacks. Two fusions:

\subsubsection{Late Fusion}\label{late-fusion}

\begin{itemize}
\tightlist
\item
  \textbf{Input to PinFM:} user sequence only.\\
\item
  \textbf{Output:} a \textbf{user embedding}.
\item
  \textbf{Crossing:} the downstream ranker crosses this user embedding
  with candidate features.
\item
  \textbf{Pros:} easy to cache per request (identical for all
  candidates).
\item
  \textbf{Cons:} cannot contextualize the user history \textbf{to each
  candidate}.
\end{itemize}

\subsubsection{Early Fusion (default)}\label{early-fusion-default}

\begin{itemize}
\tightlist
\item
  \textbf{Input to PinFM:} user sequence \textbf{plus the candidate
  token appended}.
\item
  \textbf{Outputs:} (i) a \textbf{user embedding}, and (ii) a
  \textbf{crossed user×candidate embedding} learned inside the
  transformer.
\item
  \textbf{Pros:} stronger prediction due to candidate-conditioned
  encoding.
\item
  \textbf{Caching:} still efficient via \textbf{DCAT} (see below).
\item
  \textbf{Extras:} inject candidate features (e.g., GraphSAGE) by
  \textbf{summing} them with the candidate-ID embedding and adding a
  \textbf{small alignment loss}.
\end{itemize}

\textbf{Terminology.} - \textbf{Contextual (candidate-token output):}
``how this candidate fits this user at this moment.'' -
\textbf{Intrinsic (pretrained ID embedding):} ``what the candidate is,
regardless of user.''

\textbf{Training notes.} Early fusion is default; optionally
\textbf{keep NTL/MTL during fine-tune}, set the PinFM learning rate (
\approx \tfrac{1}{10} ) of the ranker's, add \textbf{ranking loss on
PinFM outputs}, and use an \textbf{MSE alignment} to the final
prediction.

\subsubsection{Input-sequence construction (during
fine-tuning)}\label{input-sequence-construction-during-fine-tuning}

\begin{itemize}
\tightlist
\item
  \textbf{User context tokens:} the most recent ( L\_d ) actions (e.g.,
  ( L\_d=20 )).
\item
  \textbf{Candidate token:} appended for early fusion.
\item
  \textbf{Optional features:} sum content/graph embeddings into the
  candidate token; optionally add a small \textbf{Learnable Token (LT)}
  before the candidate to capture extra user×candidate interactions.
\end{itemize}

\begin{center}\rule{0.5\linewidth}{0.5pt}\end{center}

\subsection{Cold-start robustness}\label{cold-start-robustness}

\textbf{Problem.} ID-heavy models underperform on fresh items with
little ID history.

\textbf{Mitigations.}

\begin{enumerate}
\def\labelenumi{\arabic{enumi}.}
\item
  \textbf{Candidate-ID Randomization (\textasciitilde10\%)}\\
  Randomize the candidate ID about 10\% of the time during fine-tune to
  simulate unseen IDs and force reliance on \textbf{sequence context}
  and \textbf{non-ID features}.
\item
  \textbf{Item-Age-Dependent Dropout}\\
  Apply stronger dropout on PinFM outputs for fresh items, right before
  feature crossing:

  ```python if age\_days \textless{} 7: out = Dropout(p=0.7)(out) elif
  age\_days \textless{} 28: out = Dropout(p=0.5)(out)
\end{enumerate}




\end{document}
